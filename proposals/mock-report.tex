\documentclass[11pt,a4paper]{scrartcl}
\usepackage[utf8]{inputenc}
\usepackage[T1]{fontenc}
\usepackage[english]{babel}
\usepackage{ragged2e}
\usepackage{hyperref}
\usepackage{enumitem}
\usepackage{booktabs}
\usepackage{geometry}
\usepackage{fancyhdr}
\usepackage{longtable}
\usepackage{array}
\usepackage{amssymb}
\usepackage{mdframed}
\geometry{a4paper, margin=2.5cm}
\setlength{\emergencystretch}{3em}

\mdfdefinestyle{proposalframe}{%
    linewidth=0.8pt,
    innerleftmargin=10pt,
    innerrightmargin=10pt,
    innertopmargin=8pt,
    innerbottommargin=8pt,
    roundcorner=3pt
}

\pagestyle{fancy}
\fancyhf{}
\fancyhead[L]{\small CAPRA --- Software Engineering Audit}
\fancyhead[R]{\small \today}
\fancyfoot[C]{\thepage}

\title{Software Engineering Audit Report\\[0.3em]\large Relazione\_Ingegneria\_Software-1.pdf\\[0.8em]\normalsize\textit{--- MOCK VERSION for layout review ---}}
\author{CAPRA --- Automated Audit System}
\date{\today}
\begin{document}
\maketitle
\tableofcontents
\newpage

% ══════════════════════════════════════════════════════════════
% SECTION 1: PROJECT OVERVIEW (renamed from Summary)
% ══════════════════════════════════════════════════════════════
\section{Project Overview}

The project implements a \textbf{library management system} designed to handle book cataloging, user registration, and loan operations for a university library. Users can register, browse the catalog, search by title/author/genre, request book loans, extend active loans, and post comments on works and volumes. Librarians manage the catalog (add, edit, remove books) and process loan conclusions. The system is built with \textbf{Java} using \textbf{JavaFX} for the graphical interface and \textbf{PostgreSQL} for data persistence, following an MVC architectural pattern with DAO-based data access.

\medskip
\noindent\rule{\textwidth}{0.4pt}\\
{\small Issues found: \textbf{6} (HIGH: 1, MEDIUM: 4, LOW: 1) \quad---\quad Features: 6/7 present, 1 partial}

\newpage

% ══════════════════════════════════════════════════════════════
% SECTION 2: REQUIREMENTS ANALYSIS --- 3 PROPOSALS
% ══════════════════════════════════════════════════════════════
\section{Requirements Analysis --- 3 Proposals}

\textit{This section lists the requirements identified in the document and indicates whether each has an associated use case. Requirements without a linked UC need attention.}

% ──────────────────────────────────────────────────
\subsection{Proposal A --- Two subsections with description lists}
% ──────────────────────────────────────────────────

\begin{mdframed}[style=proposalframe]

\subsubsection*{Requirements with Associated Use Cases}
The following requirements are linked to at least one use case.

\begin{description}[style=nextline, leftmargin=1em, font=\normalfont\bfseries]
  \item[RF-1 --- Registrazione utente]
  $\rightarrow$ UC-1 (Registrazione)

  \item[RF-2 --- Login utente]
  $\rightarrow$ UC-0 (Login)

  \item[RF-3 --- Gestione prestiti]
  $\rightarrow$ UC-2 (Effettua prestito), UC-2.1 (Cancella prestito), UC-2.2 (Prolunga prestito), UC-2.3 (Visualizza prestiti), UC-4 (Conclude Prestito)

  \item[RF-4 --- Gestione catalogo libri]
  $\rightarrow$ UC-3 (Aggiunge libro), UC-3.1 (Modifica libro), UC-3.2 (Rimuove libro), UC-5 (Visualizza catalogo)

  \item[RF-5 --- Ricerca libri]
  $\rightarrow$ UC-5.1 (Cerca per titolo/autore/genere), UC-5.2 (Verifica disponibilit\`a)

  \item[RF-6 --- Gestione commenti]
  $\rightarrow$ UC-6 (Commento su opera), UC-6.1 (Commento su volume), UC-6.2, UC-6.3 (Visualizza commenti)

  \item[RF-7 --- Gestione profilo]
  $\rightarrow$ UC-0.1 (Modifica account), UC-0.2 (Elimina account)

  \item[RF-8 --- Limite massimo prestiti]
  $\rightarrow$ UC-2.4 (Limite massimo di prestiti)
\end{description}

\subsubsection*{Requirements without Associated Use Cases}
The following requirements lack a formally linked use case.

\begin{description}[style=nextline, leftmargin=1em, font=\normalfont\bfseries]
  \item[RF-9 --- Gestione notifiche scadenza]
  No use case references an automated notification system for overdue loans.\\
  \textbf{Suggestion:} Define a UC for loan expiry notification or remove the requirement.

  \item[RF-10 --- Report statistiche biblioteca]
  No use case covers statistical reporting for the librarian.\\
  \textbf{Suggestion:} Add a UC for generating library reports or clarify this is out of scope.
\end{description}

\end{mdframed}

% ──────────────────────────────────────────────────
\subsection{Proposal B --- Compact table + details for gaps}
% ──────────────────────────────────────────────────

\begin{mdframed}[style=proposalframe]

\begin{center}
\begin{longtable}{l >{\RaggedRight\arraybackslash}p{5cm} c >{\RaggedRight\arraybackslash}p{4cm}}
\toprule
\textbf{Req.} & \textbf{Name} & \textbf{UC?} & \textbf{Linked UCs} \\
\midrule
\endhead
RF-1 & Registrazione utente & $\checkmark$ & UC-1 \\
RF-2 & Login utente & $\checkmark$ & UC-0 \\
RF-3 & Gestione prestiti & $\checkmark$ & UC-2, UC-2.1, UC-2.2, UC-2.3, UC-4 \\
RF-4 & Gestione catalogo & $\checkmark$ & UC-3, UC-3.1, UC-3.2, UC-5 \\
RF-5 & Ricerca libri & $\checkmark$ & UC-5.1, UC-5.2 \\
RF-6 & Gestione commenti & $\checkmark$ & UC-6, UC-6.1, UC-6.2, UC-6.3 \\
RF-7 & Gestione profilo & $\checkmark$ & UC-0.1, UC-0.2 \\
RF-8 & Limite max prestiti & $\checkmark$ & UC-2.4 \\
\midrule
RF-9 & Notifiche scadenza & $\times$ & --- \\
RF-10 & Report statistiche & $\times$ & --- \\
\bottomrule
\end{longtable}
\end{center}

\paragraph{Requirements without UCs:}

\begin{itemize}[leftmargin=*]
  \item \textbf{RF-9 --- Notifiche scadenza:} No UC references automated notifications. $\rightarrow$ Define a UC or mark as out of scope.
  \item \textbf{RF-10 --- Report statistiche:} No UC covers reporting. $\rightarrow$ Add UC or clarify scope.
\end{itemize}

\end{mdframed}

% ──────────────────────────────────────────────────
\subsection{Proposal C --- Minimal list, only gap details}
% ──────────────────────────────────────────────────

\begin{mdframed}[style=proposalframe]

The document specifies \textbf{10 functional requirements}. Of these, \textbf{8} are linked to at least one use case and \textbf{2} lack a formal UC definition.

\subsubsection*{Requirements Linked to Use Cases}
\begin{itemize}[nosep]
  \item RF-1 (Registrazione) $\rightarrow$ UC-1
  \item RF-2 (Login) $\rightarrow$ UC-0
  \item RF-3 (Gestione prestiti) $\rightarrow$ UC-2, UC-2.1, UC-2.2, UC-2.3, UC-4
  \item RF-4 (Gestione catalogo) $\rightarrow$ UC-3, UC-3.1, UC-3.2, UC-5
  \item RF-5 (Ricerca libri) $\rightarrow$ UC-5.1, UC-5.2
  \item RF-6 (Gestione commenti) $\rightarrow$ UC-6, UC-6.1, UC-6.2, UC-6.3
  \item RF-7 (Gestione profilo) $\rightarrow$ UC-0.1, UC-0.2
  \item RF-8 (Limite max prestiti) $\rightarrow$ UC-2.4
\end{itemize}

\subsubsection*{Requirements without Use Cases}
\begin{itemize}[leftmargin=*]
  \item \textbf{RF-9 --- Notifiche scadenza:} Mentioned in domain model but no UC formalizes notification behavior. $\rightarrow$ Add a dedicated UC.
  \item \textbf{RF-10 --- Report statistiche:} Referenced informally but not modeled. $\rightarrow$ Add UC or remove from scope.
\end{itemize}

\end{mdframed}

\newpage

% ══════════════════════════════════════════════════════════════
% SECTION 3: ARCHITECTURE ANALYSIS (unchanged from current)
% ══════════════════════════════════════════════════════════════
\section{Architecture Analysis}

The project follows a \textbf{layered Model-View-Controller (MVC)} architecture. The presentation layer uses \textbf{JavaFX} with dedicated controllers per view; the business logic layer processes operations and validation; the persistence layer uses \textbf{DAO pattern} with \textbf{PostgreSQL}. The \textbf{Singleton} pattern manages the database connection. Components connect hierarchically: Controllers $\rightarrow$ Services $\rightarrow$ DAOs $\rightarrow$ Database.

No significant architecture issues were identified.


% ══════════════════════════════════════════════════════════════
% SECTION 4: USE CASE ANALYSIS (revised: internal problems only,
%            split by template/no template, ALL UCs shown)
% ══════════════════════════════════════════════════════════════
\section{Use Case Analysis}

The document describes \textbf{27 use cases}. This section analyzes the \emph{internal quality} of each use case (completeness, clarity, consistency). Traceability to requirements and tests is analyzed in subsequent sections.

\subsection{Use Cases with Structured Template}
The following use cases have a formal template (actor, preconditions, flows, etc.).

\subsubsection*{UC-1 --- Registrazione}
$\checkmark$ No internal issues identified. Template is complete.

\subsubsection*{UC-2 --- Effettua prestito}
\begin{description}[style=nextline, leftmargin=1em]
  \item[UC-STRUCT-001 --- \fbox{\textsc{medium}}]
  \textbf{Problem:} The business rule ``maximum 3 concurrent loans'' is mentioned in the domain model but not reflected in the UC template's preconditions or alternative flows.

  \textbf{Suggestion:} Add a precondition ``User has fewer than 3 active loans'' and an alternative flow describing the rejection when the limit is exceeded.
\end{description}

\subsubsection*{UC-3 --- Aggiunge libro}
\begin{description}[style=nextline, leftmargin=1em]
  \item[UC-STRUCT-002 --- \fbox{\textsc{medium}}]
  \textbf{Problem:} The actor is named ``Biblioteca'' in the UC diagram but ``Bibliotecario'' in the template text.

  \textbf{Suggestion:} Unify actor naming: choose one term and apply it consistently throughout the document.
\end{description}

\subsubsection*{UC-4 --- Conclude Prestito}
\begin{description}[style=nextline, leftmargin=1em]
  \item[UC-STRUCT-002 --- \fbox{\textsc{medium}}]
  \textbf{Problem:} Same actor naming inconsistency as UC-3.

  \textbf{Suggestion:} Use the same unified actor name chosen for UC-3.
\end{description}

\subsection{Use Cases without Structured Template}
The following use cases are referenced in diagrams or narrative but lack a formal template description. This is noted as an observation; a missing template does not necessarily indicate an error if the UC is sufficiently simple.

\begin{itemize}[nosep]
  \item UC-0 --- Login
  \item UC-0.1 --- Modifica account
  \item UC-0.2 --- Elimina account
  \item UC-2.1 --- Cancella prestito
  \item UC-2.2 --- Prolunga prestito
  \item UC-2.3 --- Visualizza prestiti
  \item UC-2.4 --- Limite massimo di prestiti
  \item UC-3.1 --- Modifica libro
  \item UC-3.2 --- Rimuove libro
  \item UC-3.3 --- CRUD libro (summary)
  \item UC-5 --- Visualizza catalogo
  \item UC-5.1 --- Cerca libro
  \item UC-5.2 --- Verifica disponibilit\`a opera
  \item UC-5.3 --- Recupero ISBN
  \item UC-6 --- Commento su opera
  \item UC-6.1 --- Commento su volume
  \item UC-6.2 --- Visualizza commenti opera
  \item UC-6.3 --- Visualizza commenti volume
  \item UC-7 --- Login Biblioteca
  \item UC-8 --- Visualizza catalogo Biblioteca
  \item UC-9 --- Validazione campi
  \item UC-10 --- Gestione errori login
  \item UC-11 --- Gestione errori prenotazione
\end{itemize}


% ══════════════════════════════════════════════════════════════
% SECTION 5: TESTING ANALYSIS (unchanged from current)
% ══════════════════════════════════════════════════════════════
\section{Testing Analysis}

\subsection{Testing Strategy}
The project uses \textbf{JUnit} for unit tests and \textbf{TestFX} for UI tests. Unit tests cover core business operations (loan reservations, cancellations, ISBN retrieval, availability checks). UI tests verify login flows, field validation, and book reservation interactions. Error scenarios include invalid logins, field validation failures, unavailable books, and exceeded loan limits.

\subsection{Missing Test Coverage}
The following use cases lack explicit test coverage:
\begin{itemize}[nosep]
  \item UC-0.1 --- Modifica account: no unit or UI test found
  \item UC-0.2 --- Elimina account: no unit or UI test found
  \item UC-2.3 --- Visualizza prestiti: no unit or UI test found
  \item UC-2.4 --- Limite massimo di prestiti: no unit or UI test found
  \item UC-3 --- Aggiunge libro: no unit or UI test found
  \item UC-3.1 --- Modifica libro: no unit or UI test found
  \item UC-3.2 --- Rimuove libro: no unit or UI test found
  \item UC-3.3 --- CRUD libro: no unit or UI test found
  \item UC-4 --- Conclude Prestito: no unit or UI test found
  \item UC-5 --- Visualizza catalogo: no unit or UI test found
  \item UC-5.1 --- Cerca libro: no unit or UI test found
  \item UC-6 --- Commento su opera: no unit or UI test found
  \item UC-6.1 --- Commento su volume: no unit or UI test found
  \item UC-6.2 --- Visualizza commenti opera: no unit or UI test found
  \item UC-6.3 --- Visualizza commenti volume: no unit or UI test found
  \item UC-7 --- Login Biblioteca: no unit or UI test found
  \item UC-8 --- Visualizza catalogo Biblioteca: no unit or UI test found
\end{itemize}

\textbf{Suggestion:} Add at least one test per uncovered use case, starting from the most critical business operations.


% ══════════════════════════════════════════════════════════════
% SECTION 6: TRACEABILITY ANALYSIS (moved from Requirements)
% ══════════════════════════════════════════════════════════════
\section{Traceability Analysis}

This section maps each use case to its design implementation and test coverage to identify traceability gaps. The traceability analysis runs independently from the issue-detection agents.

Of \textbf{27} traced use cases: 10 fully covered, 17 with gaps (all missing test coverage).

\subsection{Traceability Gaps}
\begin{description}[style=nextline, leftmargin=1em]
  \item[UC-0.1 --- Modifica account]
  Design: $\checkmark$ ProfiloController \quad Test: $\times$ No test found

  \textbf{Suggestion:} Add a test verifying account modification.

  \item[UC-0.2 --- Elimina account]
  Design: $\checkmark$ ProfiloController.cancellaUtente() \quad Test: $\times$ No test found

  \textbf{Suggestion:} Add a test for account deletion flow.

  \item[\textit{[...remaining 15 gaps follow same pattern...]}]
\end{description}

\subsection{Fully Traced Use Cases}
\begin{itemize}[nosep]
  \item UC-0 --- Login
  \item UC-1 --- Registrazione
  \item UC-2 --- Effettua prestito
  \item UC-2.1 --- Cancella prestito
  \item UC-2.2 --- Prolunga prestito
  \item UC-5.2 --- Verifica disponibilit\`a opera
  \item UC-5.3 --- Recupero ISBN
  \item UC-9 --- Validazione campi
  \item UC-10 --- Gestione errori login
\end{itemize}


% ══════════════════════════════════════════════════════════════
% SECTION 7: MISSING FEATURES (unchanged)
% ══════════════════════════════════════════════════════════════
\section{Missing Features}

\begin{center}
\begin{tabular}{>{\RaggedRight\arraybackslash}p{6cm} c c}
\toprule
\textbf{Feature} & \textbf{Status} & \textbf{Coverage} \\
\midrule
Definition and Documentation of Use Cases & \textsc{partial} & 60\% \\
\textit{6 other feature(s)} & \textsc{present} & 100\% \\
\bottomrule
\end{tabular}
\end{center}

\paragraph{Definition and Documentation of Use Cases}
Templates lack pre/post-conditions. $\rightarrow$ Add formal conditions to each UC template.


\end{document}
